%% 中文摘要页
\begin{ChineseAbstract}
论文中文摘要字数约为300$\textasciitilde$600字,如遇特殊需要字数可以略多,限一页。

论文摘要是论文内容不加注释和评论的简短陈述,一般以第三人称语气写成。摘要的编写应遵循下列原则:
\begin{enumerate}[label={\arabic*)},itemindent=3.5em,leftmargin=0em] 
\item 摘要应具有独立性和自含性,即不阅读论文的全文,就能获得必要的信息。摘要是学位论文的缩影,是学位论文的主要内容、见解、结论简短明了的缩写。
\item 摘要应是一篇完整的短文,可以独立使用,可以引用。
\item 摘要的内容应包含与论文等同量的主要信息,供读者确定有无必要阅读全文,也可供文摘汇编等二次文献采用。
\item 摘要一般应说明研究工作的目的意义、研究方法、研究结果、主要结论及意义、创造性成果和新见解,而重点是结论和创新点。
\item 要用文字表达,不要附图和照片,除了实在无变通办法可用以外,摘要中不用图、表、化学结构式、非公知公用的符号和术语,不要使用表格、公式、上下标以及其他特殊符号,要突出重点,阐述清楚,少用数据表。论文摘要用语力求简洁、准确。原则上300$\textasciitilde$600字。
\end{enumerate}

\ChineseKeywords{摘要;论文;要求;字数;格式(关键词个数为3$\textasciitilde$5个,正式写作请删除此括号)}
在工业控制中心等复杂生产环境中,传统基于鼠标与触控的人机交互方式难以满足多任务、高频率与高安全性的操作需求,语音交互因其自然性与低操作负担成为重要发展方向。然而,工业语音指令普遍存在专业术语密集、空间与方位描述模糊以及强场景依赖等特点,导致通用语音识别与语义理解模型在工业场景中的适用性受限。针对上述问题,本文围绕“工业语音理解与模糊指令解析”这一核心目标,构建了一套融合语音识别、语义理解与知识图谱推理的端到端工业语音交互框架。
首先,针对缺乏工业领域语音数据的问题,本文自主构建了面向钢铁生产控制场景的工业语音与语义语料体系,系统完成了语音脚本设计、多条件录制、数据清洗与增强以及语义标注规范制定等工作,为模型训练与评估提供了可靠的数据基础。在此基础上,基于 FunASR 框架对语音识别模型进行领域自适应微调,有效提升了专业词汇与设备名称的识别准确率。语义理解层采用 BERT 与槽位填充联合建模方式,实现对操作意图、设备实体与方向信息的结构化抽取,在自建语义数据集上取得了较高的识别精度。
针对工业语音中大量存在的模糊方位与相对位置描述,本文引入工业知识图谱作为场景知识表达载体,并提出一种由链路预测驱动的模糊指令解析机制,将语言层语义与设备间空间拓扑关系进行融合推理。为解决现有链路预测模型在多关系工业知识图谱中存在的语义混叠与特征冗余问题,本文在 SAttLE 模型基础上提出 DCI-Link 模型,通过正交子空间投影实现语义解耦,引入稀疏门控机制实现子空间选择性激活,并采用 GateFuse 融合结构增强特征稳定性,从而提升关系推理的判别能力与可解释性。
在系统层面,本文完成了工业语音交互原型系统设计与实现,实现了从语音输入、语义解析、知识推理到监控界面切换的完整闭环流程,并通过与工业监控平台接口对接验证了系统的可落地性。实验结果与系统验证表明,所构建的多层语义融合框架能够有效提升工业语音交互系统对模糊指令的理解与执行能力。本文研究为面向复杂工业场景的智能语音交互与语义驱动控制提供了一种具有工程可行性与理论价值的技术路径
\end{ChineseAbstract}

%% 英文摘要页
\begin{EnglishAbstract}

   In environmental economics, environmental resources including environmental quality are categorized as amenity resources. Due to its importance to human welfare, the amenity resources theoretical study and valuation is an ongoing issue at the academic frontier in the environmental economics area.  
  
\EnglishKeywords{Key word 1; Key word 2; Key word 3; $\cdots$}
注:论文的英文摘要应有英文题目和关键词,内容与中文摘要相同,用另页置于中文摘要之后;外文摘要实词在300个左右。


\end{EnglishAbstract}
