\chapter{引言(绪论)}

\section{研究背景}

本部分主要介绍选题的研究背景,根据实际情况自行填写。

自行确定是否使用三级或四级标题。宋体小四号,两端对齐,首行缩进2字符,段前0.1行,段后0.1行,1.3倍行距(段落中有数学表达式时,可根据需要设置该段的行距)。引文内容可用楷体。论文中出现英文时需要使用Times New Roman字体。这个结构可以根据自己论文需要进行调整。这些文字都是编的,根据需要自己撰写。

环境中黑炭(black carbon)气溶胶的主要来源包括各种化石燃料和生物质燃料的不完全燃烧过程,这些不完全燃烧在自然界和人类活动中都会发生,因此,环境中黑炭气溶胶的来源十分广泛\cite{usa1}。对当今大气环境中的黑炭,其主要来源是人类相关的燃料燃烧活动,此外,一些自然过程也会产生黑炭,如森林火灾、草原火灾等\cite{wang2,3}。据过去的排放清单研究,大气环境中黑炭气溶胶的来源主要包括:1)有机燃料的燃烧,主要包括能源行业、工业部门、交通运输行业、居民生活中煤、石油、天然气和各种生物质燃料的使用。通常而言,燃烧效率越高,产生的黑炭气溶胶的量越低;2)工业炼焦,主要包括炼焦过程中的炼制过程、焦炉加热系统以及焦炉煤气的泄漏等等 ;3)工业制砖,主要包括制砖过程中物料破碎输送、坯体人工干燥和烧成工段等过程;4)垃圾焚烧,包括生活垃圾和工业废料的燃烧过程;5)天然火灾和野外农业废弃物燃烧,包括森林、草原火灾和秸秆的燃烧。目前大部分研究表明,民用取暖和做饭过程中的燃料燃烧和城市柴油车是黑炭气溶胶大气排放量最大的源\cite{4,5,6,7}。这些文字都是编的,根据需要自己撰写。
\footnote{ 论文正文中的脚注,宋体小五号,英:Time New Roman,两端对齐,单倍行距。}

 
\section{研究意义}
本部分主要介绍选题的研究意义,根据实际情况自行填写。

自行确定是否使用三级或四级标题。宋体小四号,两端对齐,首行缩进2字符,段前0.1行,段后0.1行,1.3倍行距(段落中有数学表达式时,可根据需要设置该段的行距)。引文内容可用楷体。论文中出现英文时需要使用Times New Roman字体。这个结构可以根据自己论文需要进行调整。



%\subsection{假设}
%
%\section{选题的目的意义}
%
%\subsection{选题目的}
%
%本部分主要撰写选题的目的,根据实际情况自行填写。\parencite{王文清2009CALIS}
%
%\citep{王文清2009CALIS}
%
%自行确定是否使用四级标题。论文中出现英文时需要使用Times New Roman字体。这个结构可以根据自己论文需要进行调整。这些文字都是编的,根据需要自己撰写。
%
%\subsection{选题意义}
%
%本部分主要撰写选题的意义,根据实际情况自行填写。
%
%自行确定是否使用四级标题。论文中出现英文时需要使用Times New Roman字体。这个结构可以根据自己论文需要进行调整。这些文字都是编的,根据需要自己撰写。
%
%\subsection{选题的理论价值与应用价值}
%
%本部分主要撰写选题的理论与应用价值,根据实际情况自行填写。
%
%自行确定是否使用四级标题。论文中出现英文时需要使用Times New Roman字体。这个结构可以根据自己论文需要进行调整。